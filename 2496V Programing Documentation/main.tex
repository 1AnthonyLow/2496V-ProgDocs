\documentclass{article}
\title{\textbf{Programming Documentation}}
\author{2496V}
\date{2023-2024}
\usepackage{amsmath}
\usepackage{geometry}
\geometry{a5paper, portrait, margin=0.75in}
\begin{document}
\maketitle

\newpage
\renewcommand*\contentsname{Table of Contents}
\tableofcontents

\twocolumn \section{Abstract}
This is a documentation of 2496V’s development process in creating the code we use in competitions. We have gone through days, weeks, and months of theorizing, writing, and tuning our code to get to the point where we are today. This document covers our growth from our very first day programming for this season, to our last. We formatted our document in a way which reflects the chronological order of the development of our code. We start with our goals for our robot controls and our autonomous routines, in both matches and programming skills. This section covers everything we want to accomplish in terms of programming; what abilities we want to give our robot through coding. While the mechanical aspect of our robot acts as a vessel for us to win matches and score points, nothing functions without software. We decided to make this “notebook” because we believe in the importance of code and want to showcase its power as it works behind the scenes everytime we turn on our robot.This is a documentation of 2496V’s development process in creating the code we use in competitions. We have gone through days, weeks, and months of theorizing, writing, and tuning our code to get to the point where we are today. This document covers our growth from our very first day programming for this season, to our last. We formatted our document in a way which reflects the chronological order of the development of our code. We start with our goals for our robot controls and our autonomous routines, in both matches and programming skills. This section covers everything we want to accomplish in terms of programming; what abilities we want to give our robot through coding. While the mechanical aspect of our robot acts as a vessel for us to win matches and score points, nothing functions without software. We decided to make this “notebook” because we believe in the importance of code and want to showcase its power as it works behind the scenes everytime we turn on our robot.This is\\


\onecolumn
\section{Proof}
Vieta's Formulas: \\
\indent \emph{For all quadratics $ax^2+bx+c=0$, the sum and product of the roots is defined by:}\\
\indent \indent $r+s=-b/a$\\
\indent \indent $rs=c/a$\\
\indent Proof: \\
\indent \indent1. Quadratics can also be written in root form:\,$a(x-r)(x-s)=0$\\
\indent \indent2. Expanding this form gives $ax^2-a(r+s)x+ars=0$\\
\indent \indent3. Like terms imply:\\
\indent \indent \indent 1. $-a(r+s)=b$
\[\boxed{r+s=-b/a}\\\]
\indent \indent \indent 2. $ars=c$
\[
\boxed{rs=c/a}\\
\]

\section{PID}


\section{Chassis Movement}


\section{Motion Profiling}


\section{Programming Skills}


\section{Autonomous Routines}



\end{document}